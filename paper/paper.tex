\documentclass[conference]{IEEEtran}
\IEEEoverridecommandlockouts
% The preceding line is only needed to identify funding in the first footnote. If that is unneeded, please comment it out.
\usepackage{cite}
\usepackage{amsmath,amssymb,amsfonts}
\usepackage{algorithmic}
\usepackage{graphicx}
\usepackage{textcomp}
\usepackage{xcolor}
\def\BibTeX{{\rm B\kern-.05em{\sc i\kern-.025em b}\kern-.08em
    T\kern-.1667em\lower.7ex\hbox{E}\kern-.125emX}}
\begin{document}

\title{Designing and Developing a Personalised Recommender System}

\author{\IEEEauthorblockN{Thomas Butterfield}
\IEEEauthorblockA{\textit{dept. Computer Science} \\
\textit{Durham University}\\
Durham, United Kingdom \\
thomas.butterfield@durham.ac.uk}
}

\maketitle


\section{Introduction}

\subsection{Domain of application}
The domain of this application is a recommender system for bars, using the Yelp dataset.
Decide whether I want to define the time-span for the date, 
depends on if there is too much data from all available time period.
I am also taking into accounts Covid-19 data when generating recommendations.

\subsection{Related work review}
Some related work

\subsection{Purpose/Aim}
The purpose of this application is to give suitable suggestions for a user to go to.
Using information about the user such as their location and personal preferences.


\section{Methods}

\subsection{Data description}
The data I am using is taken from the Yelp dataset,
it includes user reviews of different businesses and services in a specific location.
There are 10 cities included in the dataset: Montreal, Calgary, Toronto, Pittsburgh, Charlotte, 
Urbana-Champaign, Phoenix, Las Vegas, Madison, and Cleveland. 
These user reviews include ratings, text feedback, and other such information.

\subsection{Data preparation and feature selection}
Prepare data
Of the 209,393 businesses in the dataset, 168,903 are still open.
The entire Yelp dataset is huge and much of it is not necessary for my domain of bars.
As such I prepared the data by eliminating any data not relevant to my domain.
I selected features such as user ratings and a particular user's average rating,
since if a user typically rates places they go highly, but rates a particular bar low,
this is more significant than a user who always rates places low.

\subsection{Hybrid scheme}
Which two algorithms
A hybrid scheme is a good way to design a recommender system, since you can get the
best of both algorithms if done properly.
Meaning better recommendations than either algorithm could achieve individually.

\subsection{Recommendation techniques/algorithms}
The first recommender system is collaborative filtering.
I am using a weighted mixed combination of these two systems in order to produce the results
of my hybrid recommender system.

\subsection{Evaluation methods}
How to evaluate


\section{Implementation}

\subsection{Input interface}
The input interface is the command line.
The program offers users opportunites to input information about themselves,
as well as make choices from a selection of items which the system provides or suggests.

\subsection{Recommendation algorithm}
What algorithm

\subsection{Output interface}
The output interface is the command line.
The system can output recommendations for bars which the user might like,
as well as information about how the system works and why certain suggestions were made,
at the user's request.


\section{Evaluation results}

\subsection{Comparison against baseline implementation}
Compare vs generic suggestions

\subsection{Comparison against hybrid recommenders in related studies}
Read some papers

\subsection{Ethical issues}
People's personal data


\section{Conclusion}

\subsection{Limitations}
What can it not do?

\subsection{Further developments}
What could I do in the future \cite{5284958}. 


\bibliographystyle{IEEEtran}
\bibliography{IEEEabrv,references}


\end{document}
